%-------------------------------------------
	%	第二章
	\chapter{数据库整体设计介绍}
	
	
	%--------------------------------------------------------
	\section{系统架构概述}
	本文实现的单机KV数据库基于LSM-Tree存储引擎,采用分层架构设计,主要模块包括内存组件、磁盘存储组件和后台管理组件。系统核心目标是在保证写入吞吐量的同时,通过多级数据组织优化读取性能。

        内存层的具体实现


	%---------------------------------------------------------
	\section{文件结构}
	本模板由以下文件构成:
	\begin{itemize}
	\item \textcolor{blue}{main.tex} - \LaTeX 基本框架,你可以在此添加你需要的Package
	\item \textcolor{blue}{scuthesis.sty} - 川大毕设论文格式样式包,你不需要了解这个文件(除非本模板板式不合符你的需求)
	\item \textcolor{blue}{src/basic\_info.tex} - 定义论文作者基本信息
	\item \textcolor{blue}{src/prologue.tex} - 包含了封面、中英文摘要以及目录的定义
	\item \textcolor{blue}{src/chap*.tex} - 论文正文每章内容
	\item \textcolor{blue}{src/epilogue.tex} - 附录页
	\item \textcolor{blue}{ref/refs.bib} - 参考文献,bibtex文献库,推荐使用\href{http://jabref.sourceforge.net/}{JabRef} 维护
	\item \textcolor{blue}{src/declaration.tex} - 声明页
	\item \textcolor{blue}{src/acknowledgemen.tex} - 致谢页
	 
	\end{itemize}


	%-------------------------------------------------------
	\section{使用示例}
	首先说明一下,本教程是一篇self-contained的文章,本文章是直接编译本\LaTeX 模板得到,你可以具体参考模板源代码内容以学习如何使用。但是为了阐明脉络, 下面我将以一次完整使用的形式展示如何使用本模板。
	
	\subsection{编辑流程\label{procedure}}	% 三级标题
	\textcolor{red}{首先},填写自己的基本信息,例如姓名、学号之类,你需要打开basic\_info.tex文件将其填写进去。
	
	\textcolor{red}{其次},书写你的摘要,在prologue.tex里面是中英文摘要的定义处,你可以在其中编写摘要。假设你是\LaTeX 的老用户,你可能需要自己包含一些Package,那么你可以在main.tex中添加usepackage命令。
	
	\textcolor{red}{再者},在chap*.tex中编辑自己的正文,在epilogue.tex中编辑你的附录。
	
	\textcolor{red}{最后},在refs.bib中添加自己的参考文献,在acknowledgemen.tex中致谢。
	
	
	
	\subsubsection{四级标题}
	注意:2018年 \textit{《四川大学本科毕业论文(设计)格式和参考文献著录要求》}中目录标题不超过三级,
	所以四级标题不会在目录中显示。如需修改,在\textcolor{blue}{scuthesis.sty}中调整目录计数器。
	
	\subsection{数学环境}
	
	\begin{exmp}
		这是一个例子。
	\end{exmp}

	\begin{defn} \label{test}
		这是一个定义。
	\end{defn} 
	
	\begin{lem}
		引理
	\end{lem}
	
	\begin{thm}
		定理
	\end{thm}
	
	\begin{proof}
		Trivial
	\end{proof}
	
	\begin{cor}
		推论
	\end{cor}

	\begin{prop}
		命题
	\end{prop}
	
	
	%------------------------------------------------------
	\section{问题}
	\begin{enumerate}
	\item 中英文摘要分别不能超过一页,否则第二页的板式会有问题(由于本人精力有限,且该问题出现几率较小,目前暂未打算修复这个问题)。
	\item 所有文件必须是UTF-8编码,否则编译不能通过。
	\end{enumerate}
