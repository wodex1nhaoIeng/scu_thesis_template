%--------------------------------------------------
	%	第一章
	\chapter{综述}
	
	
	%-----------------------------------------------
	\section{研究背景与研究意义}
	大家好,本人系四川大学软件学院09级学生。由于个人习惯于在Linux下工作,一直以来便饱受没有word之不便。
	适逢大四毕业论文写作之时,本想使用\LaTeX{}\upcite{bobaru2009convergence} 来排版论文。但搜索一番,
	无奈发觉网上各类\LaTeX 模板虽然琳琅满目,却惟独没有我校本科毕业生所需之毕业设计论文模板。
	遂下决心自己制作一份本科生毕设模板,并将其共享出来,以方便大家。

	本模板格式参照教务处《四川大学本科毕业论文(设计)格式和参考文献著录要求》文件,同时以软件学院内部流传的一
	份word模板为蓝本制作而成,因此可能和别的学院毕设论文格式有所差异。如若有朋友发觉本模板板式和自己的学院不同,
	亦或者随着时间的推移本模板格式过时,你可以自行更改模板。当然,为了大家方便,请在github上新建一个branch,并
	发给我Pull Request。将你的劳动成果分享给大家,何乐而不为?
	
	
	%-------------------------------------------------------------
	\section{修订说明}
	
	\begin{enumerate}
		\item 本模板基于四川大学软件学院2009级前辈的模板上进行调整,同时参考了\textit{latex模板及使用说明0315-1.0}中的模板
		\item
		改动之处:页面边距、行间距、各级标题字体字号、各级标题占行、标题自动换行、附录添加、计算机代码格式设置、声明添加、致谢添加。以上改动按照2018版
		\textit{《四川大学本科毕业论文(设计)格式和参考文献著录要求》}
		\item 
		\LaTeX 突出的优点是模块化管理,虽然这份模板看上去文件比较多,但是容易上手。重点标注的红色文字是
		基本流程,只需按照\textcolor{red}{\nameref{procedure}}所述编辑相关文件即可
		\item
		在\textcolor{blue}{scuthesis.sty}中添加了一些注释,不建议初学者改动该文件,如有需要可以参考注释改动模板
		\item
		利用BibTex是规范且方便的。这里提供一个方法导出BibTex的参考文献格式:打开网站\textcolor{green}{\url{https://xs.dailyheadlines.cc/}},输入文献信息,找到相应的文献,点击下方的“引用”,打开对话框后点击底部的“BibTeX”即可导出
		\item
		该模板已上传至Overleaf,网址:\textcolor{green}{\url{https://overleaf.com/}}
		\item 
		最后推荐一个网站\textcolor{green}{\url{https://mathpix.com/}},可以导入图片、PDF等文件自动输出\LaTeX 代码
		
	\end{enumerate}
	
	
	